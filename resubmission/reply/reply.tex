% ****** Start of file apssamp.tex ******
%
%   This file is part of the APS files in the REVTeX 4.1 distribution.
%   Version 4.1r of REVTeX, August 2010
%
%   Copyright (c) 2009, 2010 The American Physical Society.
%
%   See the REVTeX 4 README file for restrictions and more information.
%
% TeX'ing this file requires that you have AMS-LaTeX 2.0 installed
% as well as the rest of the prerequisites for REVTeX 4.1
%
% See the REVTeX 4 README file
% It also requires running BibTeX. The commands are as follows:
%
%  1)  latex apssamp.tex
%  2)  bibtex apssamp
%  3)  latex apssamp.tex
%  4)  latex apssamp.tex
%
\documentclass[%
 manuscript,
%superscriptaddress,
%groupedaddress,
%unsortedaddress,
%runinaddress,
%frontmatterverbose, 
%preprint,
%showpacs,preprintnumbers,
%nofootinbib,
%nobibnotes,
%bibnotes,
 amsmath,amssymb,
 aps, onecolumn,
%%pra,
prl,
%rmp,
%prstab,
%prstper,
%floatfix,
]{revtex4-1}

\usepackage{graphicx}% Include figure files
\usepackage{dcolumn}% Align table columns on decimal point
\usepackage{bm}% bold math
\usepackage[colorlinks=true, citecolor=red, linkcolor=blue, urlcolor=black]{hyperref}% add hypertext capabilities
%\usepackage[mathlines]{lineno}% Enable numbering of text and display math
%\linenumbers\relax % Commence numbering lines

%\usepackage[showframe,%Uncomment any one of the following lines to test 
%%scale=0.7, marginratio={1:1, 2:3}, ignoreall,% default settings
%%text={7in,10in},centering,
%%margin=1.5in,
%%total={6.5in,8.75in}, top=1.2in, left=0.9in, includefoot,
%%height=10in,a5paper,hmargin={3cm,0.8in},
%]{geometry}

\usepackage{pgffor}

\usepackage{tikz}
\newcommand{\tikzcircle}[2][red,fill=red]{\tikz[baseline=-0.5ex]\draw[#1,radius=#2] (0,0) circle ;}%
\newcommand{\site}{\tikzcircle[fill=none]{2pt}}
\newcommand{\magnon}{\tikzcircle[fill=red]{2pt}}
\newcommand{\hole}{\tikzcircle[fill=blue]{2pt}}

\def\d{{\textit{d}}}
\newcommand*{\field}[1]{\mathbb{#1}}%

\newcommand{\bra}[1]{\langle#1\rvert}
\newcommand{\ket}[1]{\lvert#1\rangle}
\newcommand{\mean}[1]{\langle#1\rangle}

\newcommand{\Bra}[1]{\left\langle#1\left\rvert}
\newcommand{\Ket}[1]{\right\lvert#1\right\rangle}

\newcommand{\AFA}{\ket{\downarrow\uparrow\downarrow\uparrow\downarrow\uparrow\hdots}}
\newcommand{\AFB}{\ket{\uparrow\downarrow\uparrow\downarrow\uparrow\downarrow\hdots}}

\newcommand{\FMA}{\ket{\uparrow\uparrow\uparrow\uparrow\uparrow\uparrow\hdots}}
\newcommand{\FMB}{\ket{\downarrow\downarrow\downarrow\downarrow\downarrow\downarrow\hdots}}

\def\tj{{$t$--$J$}}
\def\tjz{{$t$--$J^z$}}
\def\tjlam{{$t$--$J$--$\lambda$}}
\def\square{{$\mathbb{Z}^2$}}

\begin{document}

%\preprint{APS/433-SAW}

\title{The fate of the spin polaron in the 1D antiferromagnet: \\  authors' reply
}

\author{Piotr Wrzosek$^1$}
\email{Piotr.Wrzosek@fuw.edu.pl}
\author{Adam K\l{}osi\'nski$^1$}
\author{Yao Wang$^2$}
\author{Mona Berciu$^3$}
\author{Cli\`o Efthimia Agrapidis$^1$}
\author{Krzysztof Wohlfeld$^1$}
 
\affiliation{%
$^1$Institute of Theoretical Physics, Faculty of Physics, University of Warsaw, Pasteura 5, PL-02093 Warsaw, Poland
}%

\affiliation{%
$^2$Department of Physics and Astronomy, Clemson University, Clemson, South Carolina 29631, USA
}%

\affiliation{%
$^3$Dept. of Physics \& Astronomy, University of British Columbia, Vancouver, BC, Canada
and Quantum Matter Institute, University of British Columbia, Vancouver, BC, Canada
}%

\date{\today}% It is always \today, today,
             %  but any date may be explicitly specified

\maketitle

We are very grateful to the Referee for such a detailed and careful reading of our manuscript and for raising important points that need to be addressed in this paper. In fact, inspired by remark (i) from the Referee (see below) we have decided to include in the paper some additional calculations---which, in our opinion, can show in a convincing way the relevance of the paper to the interpretation of ARPES on real materials (in particular, all quasi-1D antiferromagnetic cuprates).

In the current version  of the manuscript, we have also addressed all other points by the Referee (see detailed point-by-point discussion below). We believe that the paper provides novel results which strongly advance our understanding of the low-dimensional cuprates by showing that the physics of {\it quasi}-1D cuprates can to a large extent be understood using the same formalism as its {\it quasi}-2D counterparts. This makes us believe that the relatively `simple'  spin polaron and magnon quasiparticles are far more stable, and perhaps may be far more useful in resolving the mystery of doped 2D cuprates, than has been previously thought. Thus, we suggest that the paper can be published in Physical Review Letters without further changes.
\\

Report of Referee A -- LP17489/Wrzosek

{\it \color{blue}The manuscript "The fate of the spin polaron in the 1D
antiferromagnet" written by Piotr Wrzosek, Adam Klosinski, Yao Wang,
Mona Berciu, Clio Efthimia Agrapidis, and Krzysztof Wohlfeld discusses
the stability of spin polarons for magnon-magnon interactions in a
model based on the one-dimensional t-J model. The spin polaron picture
is valid in the range other than $\lambda=1$, which corresponds to the
t-J model, and the authors propose that this is important for the
interpretation of the ARPES spectrum.

Studies of the single-hole doped $t$-$J$ and Hubbard models are very
important and have a long history. More recently, new perspectives
have been provided by the use of cold atom systems, which attract much
attention. However, compared to the vast amount of related work on
hole-doped Mott insulators, it is unclear to me whether the authors'
work will have a particularly outstanding impact on and interest from
the broad community of readers in PRL. Although the calculations seem
plausible and the analysis correct, I would not recommend publication
at PRL unless the importance of the authors' work is clear and the
impact on the broader community is evident.}

The following are my comments and questions on the current manuscript.
\\

Referee's comment: {\it \color{blue}(i) The original point of the authors' study is to investigate the
dependence of the magnon-magnon interaction $\lambda$ on the properties
of hole-doped 1D antiferromagnets. But why would a study that varies
magnon-magnon interaction as a parameter be worth publishing in PRL?
In the conclusion part, it is written that the lack of pure
one-dimensionality in real materials affects the magnon-magnon
interaction. However, there are many other effects, and the
magnon-magnon interaction is only one of them. The authors need to
clarify the importance of studies in which only the strength of the
magnon-magnon interaction is varied as a parameter.}

Authors' reply:

We are very grateful to the Referee for raising this very important question. Indeed, one of the main points of this work is related to its application to real quasi-1D magnets, for it allows for an insight into a complex 2D problem by manipulating a much simpler 1D system. More precisely, as suggested in the previous version of the paper and now verified in detail, tuning the magnon-magnon interaction strength (and the XXZ anisotropy) in a purely 1D $t$--$J$ chain well approximates the hole motion in realistic {\it quasi}-1D cuprate antiferromagnets---and hence can nicely model the ARPES spectra of Sr$_2$CuO$_3$, SrCuO$_2$, or KCuF$_3$. In the current version we discuss this conjecture in far more detail and conclude that:

First, it is important to realise that a system consisting of a 1D $t$--$J$ chain in the staggered magnetic field can qualitatively model the hole motion (or ARPES spectra) of the quasi-1D cuprates. To this end, we point out that the best-known quasi-1D cuprates, such as Sr$_2$CuO$_3$ [1], SrCuO$_2$ [2], or KCuF$_3$ [3] have strong SU(2)-symmetric (i.e. Heisenberg) antiferromagnetic interaction within the chains and weak (antiferro- or ferro-magnetic) interaction between the spins in neighboring chains. Thus, the proper model to discuss the hole-motion in quasi-1D cuprates is the 1D $t$--$J$ model with weak interchain hoppings and magnetic exchange interactions between the chains. Interestingly, the latter model can be simplified. On one hand, to get a qualitative understanding of the hole motion in this model it is enough to consider the purely-1D motion of hole---though the magnetism has to be kept quasi-1D (the so-called `mixed-dimensional limit') [4]. On the other hand, assuming the onset of the long-range magnetic order at low temperatures (see below for the case when this condition is not fulfilled), the magnetic interactions between the chains can be treated on a mean-field level---which, irrespectively of the sign of the interchain coupling, yields a staggered magnetic field acting on the antiferromagnetic chain in which the hole moves [5-7]. 

Second, and crucially, we unambiguously show that the problem of a single hole in a 1D $t$--$J$ model with variable strength of magnon-magnon interactions (and XXZ anisotropy) is qualitatively the same (and quantitatively very similar) as the problem of the single hole in a 1D $t$--$J$ model in the staggered field with variable strength. The reason for this is that the staggered field disrupts the {\it fine balance} between the on-site magnon energy and the magnon-magnon attraction of the 1D $t$--$J$ model. (We note that, also at higher temperatures, i.e. when there is no long-range order and the staggered field cannot be used to simulate the coupling between the chains, this fine balance will {\it also} be disrupted due to the change of magnon on-site energies by the exchange interaction between the chains. Thus, all of the presented results will qualitatively carry on also to temperatures higher than the Neel temperature. We do not present such calculations, since they require exact diagonalisation of a full 2D problem, which heavily suffers from finite size effects and is beyond the scope of this work.)

To reflect the above discussion, we have accordingly modified the manuscript (see Summary of Changes). First, we have added an analytical discussion (in the supplemental material) showing a relation between the 1D $t$--$J$ model with  {\it tunable} magnon-magnon interactions and the 1D $t$--$J$ model {\it coupled} to the neighboring chains. Second, following the above result, we have obtained a figure [current Fig. 4(b)] showing the stability of the spin polaron quasiparticle in the 1D $t$--$J$ model {\it coupled} to the neighboring chains. Finally, we have also added the spectral function of the the 1D $t$--$J$ model {\it coupled} to the neighboring chains, cf. (added) Fig. 4(d). The latter figure shows that indeed the obtained spectral function is consistent with the observed ARPES spectra of the quasi-1D compounds. Altogether, this means that tuning the magnon-magnon interactions allows us to model the ARPES spectra of quasi-1D cuprates, and also that these spectra can be understood using the spin polaron picture.

Literature:

[1] K. M. Kojima {\it et al.}, Phys. Rev. Lett. {\bf 78}, 1787 (1997).

[2] M. Matsuda {\it et al.}, Phys. Rev. B {\bf 55}, R11953 (1997). 

[3] B. Lake {\it et al.}, Nat. Phys. {\bf 4}, 329 (2005).

[4] F. Grusdt, Z. Zhu, T. Shi, E. Demler, SciPost Phys. {\bf 5}, 057 (2018).

[5] H. J. Schulz, Phys. Rev. Lett. {\bf 77}, 2790 (1996).

[6] F. H. L. Essler, A. M. Tsvelik, and G. Delfino, Phys. Rev. B {\bf 56}, 11001(1997). 

[7] A. W. Sandvik, Phys. Rev. Lett. {\bf 83}, 3069 (1999).
\\

Referee's comment: {\it \color{blue}(ii) Since spin-charge separation is valid in the strongly coupled
regime of one-dimensional systems, it is an obvious consequence that
spin-charge separation is no longer valid in regions where the
magnon-magnon interaction is large. In addition, the spin-charge
separation, which is a consequence of quantum mechanics, is no longer
valid in regions where the magnon-magnon interaction is small since
these regions are classical states described by linear spin-wave
theory. The case of $\lambda=1$ is treated as a special parameter for
which spin-charge separation is possible, but how rigorous is the
authors' statement that spin-charge separation does not realize if
$\lambda$ is even slightly off from 1?}

Authors' reply:

This is a rigorous statement. The fact that the spin polaron picture is fully valid (or, equivalently, the spin-charge separation is merely an approximate picture) for any $\lambda \neq 1$ is due to the following argument: it is only at $\lambda =1$ that the string potential does not develop, for the cost of the nearest neighbor magnons is only then exactly cancelled by the nearest neighbor magnon attraction. To make this point more clear, we have accordingly modified the manuscript (see Summary of Changes).
 
The above statement can also be nicely seen in the numerical calculations. To this end, we have calculated the probabilities of observing $n$ magnons in the ground state ($c_n$) for several values of $\lambda$---including those very close to unity. The obtained behavior, different for $\lambda =1$  confirms the rigorous statement (see updated version of Fig. 4). 
\\

Referee's comment: {\it \color{blue}(iii) The authors state on page 3 that $A(k,\omega)$ is qualitatively
similar for $\lambda=0$ and 1, but in the last paragraph of page 4, they
emphasize that differences exist between them. In the end, I think
readers will be confused as to which position the authors want to
advocate. It is also mentioned that the similarity between Figs. 3(a)
and 3(b) indicates the robustness of the spin polaron picture, but
does that mean that no features of spin-charge separation appear in
Fig. 3(a)?}

Authors' reply:

This subtle, but crucial, issue has not been properly addressed in the paper. Below let us clarify this issue in three steps:
\begin{enumerate}
\item The two spectra, one at $\lambda=1$ and one at $\lambda =0$ (or at any $\lambda \neq 1$), are qualitatively the same, except for the important, but subtle, difference noted below.
\item The subtle difference between the two spectra is related to the onset of the gap between the ground and lowest excited stare in ARPES for each $k$: (i) at $\lambda =1$ there is no gap, (ii) at $\lambda = 0 $ (or at any $\lambda \neq 1$) the gap sets in. Crucially, however, this gap is small (of the order of $0.5t$) and hardly visible with the current ARPES resolution for all $k<\pi/2$ and is only substantial ($\simeq t$) and in principle observable in ARPES for the $k>\pi/2$ [as also stated be the Referee in point (vi) below].
\item While points 1. and 2. suggest that the robustness of the spin polaron picture, it should be stated that this does not invalidate the spin-charge separation picture, which is just valid based on the use of a different basis. In other words: (i) at $\lambda =1 $ the spin-charge separation is exact and the spin polaron picture is approximate, (ii) at any $\lambda \neq 1$ (which corresponds to the increasing coupling to the neighboring chains) the spin-charge separation picture is approximate (and its inappropriateness increases once $\lambda$ decreases from $\lambda: 1 \rightarrow 0$) and the spin polaron picture is exact.
\end{enumerate}

To better reflect this subtle issue, we have accordingly modified the current text of the paper (see Summary of Changes).
\\

Referee's comment: {\it \color{blue}(iv) SU(2) symmetry is present for $\lambda=1$, but how is SU(2) broken
for the other $\lambda$? SU(2) symmetry breaking has a significant
effect on the nature of the magnon energy cost produced by a hole. On
page 4, the authors state that a hole is under the influence of a
string potential as seen in the t-Jz model when the spin-charge
separation of $0<\lambda<1$ does not hold. Is the existence of the
potential related to SU(2) symmetry breaking? Can this potential
energy bind a hole?}

Authors' reply:

Indeed, once $\lambda \neq 1$, the SU(2) symmetry is broken. This is best visible by rewriting `back' the model with $\lambda \neq 1$ in the spin basis (i.e. more precisely, the basis in which the $t$--$J$ model is written). Then it turns out that besides the `usual' terms in the $t$--$J$ model, there is also an extra term which: (i) is proportional to $(\lambda -1)$, i.e. vanishes when $\lambda =1$, (ii) breaks the rotational SU(2) symmetry, for it distinguishes between the two Neel configurations (i.e. leads to different eigenenergies of the states with the `up-down-up-down-...' and the `down-up-down-up-...' configurations). The latter fact is actually a clear signature of the string potential present in the model once $\lambda \neq 1$: since the mobile hole leads to the `switching' between the two Neel configurations, the hole, due to the different energies of these two configurations, experiences the string potential.

In order to describe this issue in detail, we have added a short Supplementary Material (SM) to the paper (see Summary of Changes below).
\\

Referee's comment: {\it \color{blue}(v) The caption says there are dashed lines in Fig. 3(a) and 3(b), but
I can't find them. Please draw it more visibly.}

Authors' reply:

We are very grateful to the Referee for pointing out this issue. Indeed, we made a mistake in the previous version of the manuscript and omitted the dashed lines. The current version of the paper contains the correct figure (see Summary of Changes below).
\\

Referee's comment: {\it \color{blue}(vi) It would be easier for readers to understand the role of
magnon-magnon interactions if it is described why the difference
between $\lambda=0$ and 1 appears mainly as a difference in the
structure of $A(k,\omega)$ with $k>\pi/2$. }

Authors' reply:

Indeed, this is very good remark---we have followed the Referee's advice and accordingly modified the paper (see Summary of Changes below).
\\

Referee's comment: {\it \color{blue}(vii) While the single-hole doped t-J and Hubbard models have been
studied for a long time, they are also cutting-edge topics where
progress has been made in their understanding even more recently.
However, the authors seem to be biased in their citation of the
references. I recommend that the authors revise the references they
cite to make the importance of their research more convincing. The
same applies to citations of references on magnon-magnon interactions.}

Authors' reply:

This is also a very good point---we have followed the Referee's advice and accordingly modified the paper (see Summary of Changes below).

\newpage

\section{Summary of changes}

(1) We modify the last sentences of the abstract to accommodate for the new content of the paper and to clarify our standpoint on the relevance of our findings for the quasi-1D cuprates. The new abstract reads:

``The stability of the spin polaron quasiparticle, well established in studies of a single hole in the 2D antiferromagnets, is investigated in the 1D antiferromagnets using a $t$--$J$ model. We  perform an exact slave fermion transformation to the holon-magnon basis, and  diagonalize numerically the resulting model in the presence of a single hole. We prove that the spin polaron is stable for any strength of the magnon-magnon interactions {\it except} for the unique value of the SU(2)-symmetric 1D $t$--$J$ model. Fine-tuning to this unique value is extremely unlikely to occur in {\it quasi}-1D antiferromagnets, therefore the spin polaron is the stable quasiparticle of real 1D materials. Our results lead to a new interpretation of the ARPES spectra of {\it quasi}-1D antiferromagnets in the spin polaron language."

(2) In the third paragraph of the introduction we remove an unnecessary discussion of the excited states:

``Moreover, we find that the {\it excited states} of the 1D  antiferromagnet with a single hole are only very weakly affected by changing the strength of the magnon-magnon attraction, {\it i.e.} they do not provide much insight on whether there is a stable spin polaron."

Instead, we report here the importance of our finding for the interpretation of ARPES spectra of quasi-1D cuprates in the spin polaron language:

``In particular, we show that the staggered magnetic field present in {\it quasi}-1D antiferromagnets of real materials~\cite{Kojima1997, Matsuda1997, Lake2005} disrupts this fine balance between the on-site magnon energy and the magnon-magnon interaction of the 1D $t$--$J$ model. This makes the spin polaron quasiparticle stable in the {\it quasi}-1D cuprates and leads to the interpretation of the ARPES spectra~\cite{Sobota2021} of {\it quasi}-1D cuprates~\cite{Kim96, Kim1997, Fujisawa1999, Koitzsch2006, Kim06} in the spin polaron language."

(3) We include additional references to previous works on spin polaron and spin-charge separation in the first paragraph of the introduction.

(4) To avoid repeating ourselves, we shorten the last sentence of the introduction:

``This has important consequences for the interpretation of cold atom experiments~\cite{Bohrdt2021, Koepsell2021} or angle resolved photoemission spectroscopy (ARPES) spectra~\cite{Sobota2021} of quasi-1D antiferromagnets~\cite{Kim96, Kim06}, as discussed in our conclusions."

which now reads:

``The obtained results have important consequences reaching beyond condensed matter, {\it inter alia} in the interpretation of cold atom experiments~\cite{Bohrdt2021, Koepsell2021}."


(5) As pointed by Referee A, Fig. 3. was missing dashed lines mentioned in the caption of the figure. We redraw the figure including the missing lines. At the same time, we trim the figure by excluding the bottom panels which showed a redundant analysis of the n-th moments of the spectral function. Accordingly, we update the caption of the figure. It now reads:

``Properties of the excited state of the 1D $t$--$J$ model with a single hole as probed by the spectral function $A(k,\omega)$: (a) with the magnon-magnon interactions correctly included [$\lambda=1$ in (2)]; (b) without the magnon-magnon interactions [$\lambda=0$ in (2)]. The dashed (dotted) lines in (a-b) show the holon (spinon) dispersion relations respectively, as obtained from the spin-charge separation Ansatz~\cite{Ede97, Kim97}. The highest intensity peak at lowest energy in (b) is the spin polaron quasiparticle peak. Calculation performed on the 28~sites long periodic chain using exact diagonalization and with $J=0.4t$."

(6) In the section ``Results: excited states" we remove the following last sentence of the second paragraph:

``Below, we give an alternative explanation by comparing it to the spin polaron case." 

(7) We are more precise when referring to the features visible in Fig. 3. We change ``high momenta" to ``momenta $k> \pi/2$" and ``apart from the dispersive low-energy feature" to ``{\it apart} from the dispersive low-energy quasiparticle feature particularly pronounced for $k> \pi/2$" in the paragraphs 3 and 4 of section ``Results: excited states".

(8) As pointed out by Referee A, to avoid reader's confusion when comparing spectral functions with and without magnon interactions we remove the following discussion referring to the no longer present part of Fig. 3:

``To confirm this notion we calculate the first and second moments of the spectral function, see Fig. 3(c-d), and conclude that the functional forms guiding the behavior of both moments are qualitatively the same for $\lambda= 1$ and $\lambda =0$"

(9) To account for the Referee's comments (iii) and (vi) we restructure the last paragraph of section ``Results: excited states". We change

``The similarity between the spectra with and without magnon-magnon interactions enable us to understand the features appearing at $\omega / t \propto t |\cos k | $ in the spectral function for both $\lambda =1 $ and $\lambda = 0$, which, in the spin-charge separation picture, would be associated with the `free holons' (see~\onlinecite{Ede97, Kim97} and dashed lines of Fig. 3(a-b)). In the magnon-holon language, such features can be already understood in the Born approximation, which gives the dominant dispersion that is proportional to $t^2 |\cos k|$, as it is related to the holon propagating in a polaronic way by exciting a single magnon at a vertex $t |\cos k | $."

to

``These results enable us to give an alternative, albeit approximate, understanding of the dominant features appearing at $\omega \propto t |\cos k |$ in the spectrum at $\lambda=1$. These dispersions are well accounted for in the spin-charge separation picture as the `free' holons, cf. ~\cite{Ede97, Kim97} and dashed lines of Fig. 3(a-b). Here, based on the similarity between $\lambda =1 $ and $\lambda = 0$ spectra, we can approximately interpret the two dominant spectral features as being due to a holon propagating in a polaronic way by exciting a single magnon (Born approximation) at a vertex $t |\cos k | $."

(10) To account for comment (iv) by the Referee, in the first paragraph of section ``Interpretation \& critical magnon interactions" we explicitly point out when the SU(2) symmetry is broken in our investigations:

``[the SU(2) symmetry is broken in the model once $\lambda \neq 1$ in (2), see \cite{SM}]".

(11) We reformulate the last sentence of the first paragraph in section ``Interpretation \& critical magnon interactions":

``While we have verified this key result numerically [Fig.~4(a)], it is actually more instructive to give a physical argument which explains it, and also provides an intuitive understanding of all the results presented so far."

now reads:

``While this result can be verified using the observables used above, it is actually more instructive now to use a different observable [Fig.~4(a-c), for it provides a more intuitive and simpler understanding of the results that follow."

(12) We redraw Fig. 4. To address Referee's comment (ii), we include results for $c_n$ when $\lambda = 0.99$. To include a discussion on the relevance of this work for real material and the interpretation of ARPES results on quasi-1D cuprates, we replace repeating information on $c_n$ for $\lambda = 1$ with a newly calculated spectral function of a single hole in a t-J chain in presence of staggered magnetic field. 

Accordingly, we update the caption of Fig. 4. It now reads:

``Properties of 1D $t$--$J$ model with a single hole and with: (a) {\it modified} strength of the magnon-magnon interactions [$\lambda$ in (2)]; (b, d) {\it added} staggered magnetic field arising due to the coupling [$J_\perp / J$ in Eq. (1) in~\cite{SM}] to neighboring chains in a quasi-1D geometry, cf.~text and~\cite{SM}. Panels (a-b) show probabilities $c_n$ of finding a configuration with $n$ consecutive magnons attached to one side of the hole in the ground state of the respective model; panel (c) shows a pictorial view of a state with $n=4$ magnons attached to the left side of the hole; panel (d) shows the spectral function $A(k,\omega)$ calculated for the 1D $t$--$J$ model with added staggered magnetic field $J_\perp = 0.1J$. All data obtained using exact diagonalization on the 28~sites periodic chain using $J=0.4t$."

(13) In the last paragraph of section ``Interpretation \& critical magnon interactions" we include reference to new data presented in Fig. 4 in:

``[cf. Fig.~4(a), {\it inter alia} note the distinct behavior for $\lambda =0.99$ and $\lambda =1 $]."

(14) To address comment (i) by the Referee, we add a new section ``Relevance for real materials" before the section ``Conclusions". The new section reads:

``The existence of just one critical value of the magnon-magnon interactions [$\lambda =1$ in (2)] stabilising the spin-charge separation solution leads to an important consequence for real materials. Due to the nature of atomic wavefunctions and crystal structures, the best-known `1D'
antiferromagnetic materials (cf. Sr$_2$CuO$_3$, SrCuO$_2$, or KCuF$_3$) are solely {\it quasi}-1D~\cite{Kojima1997, Matsuda1997, Lake2005}. A precise model of these materials should include a small but finite staggered magnetic field $J_\perp$ [see the Supplementary Material~\cite{SM} for details], which originates in the magnetic coupling between the spins on neighboring chains~\cite{Schulz1996, Essler1997, Sandvik1999}.
Importantly, the single-hole dynamics in a 1D $t$--$J$ model with staggered fields is qualitatively the same (and quantitatively very similar, as discussed in the Supplementary Material~\cite{SM}) as
that in a 1D $t$--$J$ model with magnon-magnon interactions (and XXZ anisotropy). Here, the strength of the staggered field can be mapped to the strength of the magnon-magnon interactions.
The reason for this is that the staggered field disrupts the above-discussed fine balance in the strict 1D $t$--$J$ model, between the on-site magnon energy and the magnon-magnon attraction. 
In this case, the mobile hole in the quasi-1D cuprates experiences the string potential and forms 
the spin polaron, cf.~Fig~4(b). Therefore, the sensitivity to magnon interactions indicates that spin-charge separation is strictly speaking never realised in real materials.

One may wonder how to reconcile the above finding with the fact that ARPES experiments on quasi-1D materials have reported 
spin-charge separation~\cite{Kim96, Kim1997, Fujisawa1999, Koitzsch2006, Kim06}. That statement is based on the experimentally measured spectrum being similar 
to the one obtained for the 1D $t$--$J$ model, 
cf. Fig.~3(a) above \cite{Kim96, Kim1997, Fujisawa1999, Koitzsch2006, Kim06}. The salient fact is that the spectrum obtained for the 1D $t$--$J$
model with a realistic value of the staggered field $0<J_\perp \lesssim 0.1J$~\cite{SM}, cf.~Fig~4(d), is {\it almost indistinguishable}
from the one of Fig.~3(a). 
In fact, for the available finite size calculations with the numerical broadening $\delta = 0.05 t$, 
the only visible difference between the two spectra lies in an extremely faint quasiparticle feature present for $k>\pi/2$. The latter
feature cannot be observed with the current ARPES resolution, especially at high temperature and with a typically
weaker signal for $k>\pi/2$ in ARPES. Thus, we conclude that ARPES on {\it quasi}-1D
cuprates is correctly-explained using the spin polaron picture, with its dominant cosine-like features interpreted
as the holon exciting a magnon at a vertex $\propto t |\cos k|$ (see above)."

(15) We rewrite section ``Conclusions" to account for the new content of the paper and shorten it to more clearly express our message. Now this section reads:

``In this work we discussed the extent to which the concept of the spin polaron, well-known from the studies of a single hole in the 2D antiferromagnets~\cite{Shr88}, can be applied to the single hole problem in the 1D antiferromagnets. We find that {\it only} in the 1D SU(2) symmetric model the spin polaron is unstable to spin-charge separation due to the critical value of the magnon-magnon interactions. In contrast, the spin polaron quasiparticle is stable in the real {\it quasi}-1D antiferromagnets such as SrCuO$_2$, Sr$_2$CuO$_3$ or KCuF$_3$. The surprising robustness of the spin polaron leaves us with a question whether this simple picture can be used to study also the higher-dimensional highly-doped antiferromagnets beyond the collapse of the long-range order."

(16) To further address comment (i) by the Referee, we provide supplementary materials connecting the magnon interactions tuning parameter $\lambda$ with the physics of quasi-1D cuprates. There, we explicitely show how an effective staggered field induced by a small inter-chain coupling present in real cuprate materials translates to a reduction of magnon-magnon interactions. The content of the Supplementary Material reads:

\section{``The $t$--$J$ model for quasi-1D cuprates:\\ tuneable staggered magnetic field vs. tuneable magnon-magnon interaction}

In order to construct the $t$--$J$ model for quasi-1D cuprates, we make the following assumptions:

{\it First}, to get qualitative insight into the hole motion, we note that hopping between the chains can 
be neglected~\cite{Gru18b} and that the longer-range hopping is
very small for quasi-1D cuprates~\cite{Li2021}. Besides, the recently postulated strong coupling to phonons in 1D cuprates~\cite{Chen2021}, not included here, would only further 
disrupt the (mentioned below and in the main text of the paper) fine balance between the magnon-magnon interactions and the magnon on-site energy.

{\it Second}, the remaining Heisenberg exchange interaction between the chains 
can be represented as the staggered magnetic field (which, due can be obtained from the spin exchange between the chains~\cite{Schulz1996}, hence is called $J_\perp$ below and  in the main text of the paper):
\begin{equation}
	H_{J_\perp} = \frac{J_\perp}{2} \sum_{\mean{i,j}} \left[(-1)^i S^z_i + (-1)^j S^z_j\right].
	\label{eq:stag}
\end{equation}
The above term follows by assuming the onset of the long-range magnetic order at low temperatures, 
the magnetic interactions between the chains can be treated on a mean-field level---which, irrespectively of the sign of the interchain coupling,
yields a staggered magnetic field acting on the antiferromagnetic chain in which the hole moves~\cite{Schulz1996,
Essler1997, Sandvik1999}. We note that, also at higher temperatures, i.e. when there is no long-range order and the staggered field 
cannot be used to simulate the coupling between the chains, the (mentioned in the main text of the paper) 
fine balance between the magnon-magnon interactions and their onsite energies will {\it also} be disrupted due to the 
change of magnon on-site energies by the exchange interaction between the chains.
Thus, all of the presented results, obtained below will qualitatively model the quasi-1D cuprates also at temperatures higher than 
the Neel temperature. (We do not present such calculations, since they require exact diagonalisation 
of a full 2D problem, which heavily suffers from finite size effects and is beyond the scope of this work.)
Following \cite{Schulz1996} one can estimate the value of  $J_\perp$
in various quasi-1D cuprates: e.g. for KCuF$_3$ we obtain $J_\perp \approx 0.06 J$ [hence the assumed in Fig. 4(d) of the main text value $J_\perp =0.1J$, being
the upper bound of that estimate].

Now let us investigate how the additional staggered field looks like in the polaronic description already used in the main text. In order to 
do this we firstly show in detail the polaronic descritpion of the 1D $t$--$J$ model [i.e. how to go from Eq.~(1) to Eq.~(2) of the main text].
To this end, we start with a rotation of spins in one of the system's sublattices. This results in
%
\begin{equation}
	H_{\text{rot}} = -t\sum_{\mean{i,j}}\left(\tilde{c}_{i\sigma}^\dagger\tilde{c}_{j\bar{\sigma}} + H.c\right)
	+ J\sum_{\mean{i,j}}\left[\frac{1}{2}\left(S_i^+S_j^+ + S_i^-S_j^-\right) - S_i^zS_j^z - \frac{1}{4}\tilde{n}_i\tilde{n}_j\right].
\end{equation}
%
This allows for the introduction of holes and magnons according to the following transformations
%
\begin{equation}
	\begin{aligned}
	\tilde{c}_{i\uparrow}^\dag &= h_i, &\quad \tilde{c}_{i\uparrow} &= h_i^\dag (1 - a_i^\dag a_i), \\
	\tilde{c}_{i\downarrow}^\dag &= h_i a_i^\dag, &\quad \tilde{c}_{i\downarrow} &= h_i^\dag a_i,
	\end{aligned}
\end{equation}
%
\begin{equation}
	\begin{aligned}
		S_i^+ &= h_i h_i^\dag (1 - a_i^\dag a_i)a_i, &\quad S_i^z &= \left(\frac{1}{2} - a_i^\dag a_i \right) h_i h_i^\dag, \\
		S_i^- &= a_i^\dag (1 - a_i^\dag a_i) h_i h_i^\dag, &\quad \tilde{n}_i &= 1 - h_i^\dag h_i = h_i h_i^\dag,
	\end{aligned}
\end{equation}
where $a_i^\dag$ are bosonic creation operation at site $i$ denoting magnons and $h_i^\dag$ are fermionic creation operators at site $i$ denoting holons.
Here magnons can be understood as deviations from the state that has all the spins pointing up after the applied sublattice rotation. In the end, the 1D $t$--$J$ model (up to a shift by a constant energy) reads:
%
\begin{align}
	\mathcal{H} &= \mathcal{H}_{t} + \mathcal{H}_{J},
\end{align}
%
where,
%	
\begin{equation}
	\begin{split}
	\mathcal{H}_{t} &= t \sum_{\mean{i,j}} \left\{h_i^\dag h_j \left[ a_i + a_j^\dag (1 -  a_i^\dag a_i) \right] + h_j^\dag h_i \left[ a_j + a_i^\dag (1 -  a_j^\dag a_j) \right]\right\},
	\end{split}
	\label{eq:ht}
\end{equation}
%
\begin{equation}
	\begin{aligned}
	\mathcal{H}_{J} &= \frac{J}{2}\sum_{\mean{i,j}} h_i h_i^\dag \left[(1 - a_i^\dag a_i)(1 - a_j^\dag a_j)a_i a_j + a_i^\dag a_j^\dag (1 - a_i^\dag a_i)(1 - a_j^\dag a_j) \right] h_j h_j^\dag \\
	&+ \frac{J}{2} \sum_{\mean{i,j}} h_i h_i^\dag \left(a_i^\dag a_i + a_j^\dag a_j - 2 a_i^\dag a_i a_j^\dag a_j - 1\right) h_j h_j^\dag.
	\end{aligned}
	\label{eq:hj}
\end{equation}

Now let us investigate the staggered magnetic field term given by Eq.~\eqref{eq:stag} above.
%
Performing the same set of transformations we obtain (up to a constant energy shift),
%
\begin{equation}
	\mathcal{H}_{J_\perp} = \frac{J_\perp}{2} \sum_{\mean{i,j}} 
	\left(a_i^\dag a_i h_i h_i^\dag + a_j^\dag a_j h_j h_j^\dag \right) \approx \frac{J_\perp}{2} \sum_{\mean{i,j}} 
	h_i h_i^\dag \left(a_i^\dag a_i + a_j^\dag a_j \right) h_j h_j^\dag.
\end{equation}
%
The omitted terms on the right hand side of the approximation modify the magnetic field only around the hole and they are $\propto J_\perp \left( a_i^\dag a_i h_j h_j^\dag + a_j^\dag a_j h_i h_i^\dag \right)$. In the end, we obtain for the spin part of the Hamiltonian [$\mathcal{H}_t$ is not affected, i.e. given by Eq.~\eqref{eq:ht} above]
%
\begin{equation}
	\begin{aligned}
	\mathcal{H}_{J+J_\perp} &\equiv \mathcal{H}_{J} + \mathcal{H}_{J_\perp} \\
	&\approx \frac{J}{2}\sum_{\mean{i,j}} h_i h_i^\dag \left[(1 - a_i^\dag a_i)(1 - a_j^\dag a_j)a_i a_j + a_i^\dag a_j^\dag (1 - a_i^\dag a_i)(1 - a_j^\dag a_j) \right] h_j h_j^\dag \\
	&+ \frac{J}{2} \sum_{\mean{i,j}} h_i h_i^\dag \left[\left(1+\frac{J_\perp}{J}\right)\left(a_i^\dag a_i + a_j^\dag a_j\right) - 2 a_i^\dag a_i a_j^\dag a_j - 1\right] h_j h_j^\dag.
	\end{aligned}
\end{equation}
%
Let us introduce the XXZ anisotropy 
\begin{align}
\Delta = \frac{J_\perp}{J} 
\end{align}
and the  rescaled magnon-magnon interaction parameter 
\begin{align}
\lambda = \frac{1} {1+\Delta}.
\end{align}
Then, in the single hole limit, we can write
%
\begin{equation}
	\begin{aligned}
	\mathcal{H}_{J+J_\perp} 
	&\approx \frac{J}{2}\sum_{\mean{i,j}} h_i h_i^\dag \left[(1 - a_i^\dag a_i)(1 - a_j^\dag a_j)a_i a_j + a_i^\dag a_j^\dag (1 - a_i^\dag a_i)(1 - a_j^\dag a_j) \right] h_j h_j^\dag  \\
	&+ (1+\Delta) \frac{J}{2} \sum_{\mean{i,j}} h_i h_i^\dag \left(a_i^\dag a_i + a_j^\dag a_j - 2\lambda a_i^\dag a_i a_j^\dag a_j \right) h_j h_j^\dag.
	\end{aligned}
\end{equation}
%
Thus, once  $J_\perp \neq 0$ the final model is the $t$--$J$ model with the XXZ anisotropy $\Delta$ 
{\it and} rescaled magnon-magnon interaction $\lambda$. In TABLE~\ref{tab:params}. 
we present the values of $\lambda$, $\Delta$ calculated for the corresponding values of $J_\perp$ 
used in calculations for Fig.~4(b) and 4(d) in the main text.

\begin{table}[t!]
	\begin{center}
	\begin{tabular}{|| c || c | c ||} 
		\hline
		 ~~$J_\perp / J$~~ & ~~$\Delta$~~ & ~~$\lambda$~~ \\
		\hline\hline
		 ~~$0.01$~~ & ~~$0.01$~~ & ~~$\frac{100}{101}$~~ \\  
		\hline
		 $0.1$ & $0.1$ & $\frac{10}{11}$ \\ 
		\hline
		$0.5$ & $0.5$ & $\frac{2}{3}$ \\
		\hline
	\end{tabular}
	\end{center}
	\caption{Table presenting the relation between the value of the staggered field $J_\perp$ in the quasi-1D $t$--$J$ model and the $t$--$J$ model 
	with rescaled magnon-magnon interaction $\lambda$ and the XXZ anisotropy $\Delta$.}
	\label{tab:params}
\end{table}

\section{SU(2) symmetry breaking in the $t$--$J$ model \\ with tuneable magnon-magnon interactions}

We start by re-expressing the magnon-magnon interaction term in the `standard' (i.e. spin) language,

\begin{equation}
    a_i^\dag a_i a_j^\dag a_j = -S_i^z S_j^z + \frac{1}{4}\tilde{n}_i\tilde{n}_j - \frac{1}{2}\left(\xi_i^\mathcal{A} S_i^z \ + \xi_j^\mathcal{A} S_j^z \right)\tilde{n}_i\tilde{n}_j,
\end{equation}
%
where $\xi_i^\mathcal{A}$ equals $-1$ for $i\in\mathcal{A}$ and $1$ otherwise, with $\mathcal{A},\mathcal{B}$ denoting the two sublattices of the bipartite lattice. Thus, Hamiltonian (2) of the main text (i.e. the $t$--$J$ model with tuneable magnon-magnon interactions) reads,
%
\begin{equation}
        \begin{aligned}
    	&H = -t\sum_{\mean{i,j}}\left(\tilde{c}_{i\sigma}^\dagger\tilde{c}_{j\sigma} + \text{H.c.}\right)
	+ J\sum_{\mean{i,j}}\left\{S_i S_j - \frac{1}{4}\tilde{n}_i\tilde{n}_j 
	+ \left(\lambda-1\right) \left[S_i^z S_j^z - \frac{1}{4}\tilde{n}_i\tilde{n}_j + \frac{1}{2}\left(\xi_i^\mathcal{A} S_i^z \ + \xi_j^\mathcal{A} S_j^z \right)\tilde{n}_i\tilde{n}_j\right] \right\}.
	\end{aligned}
	\label{eq:lambda_spin_model}
\end{equation}
In the above Hamiltonian \eqref{eq:lambda_spin_model}, the term
%
\begin{equation}
    \frac{1}{2}\left(\xi_i^\mathcal{A} S_i^z \ + \xi_j^\mathcal{A} S_j^z \right)\tilde{n}_i\tilde{n}_j
        \label{eq:staggered_term}
\end{equation}
%
can be understood as a staggered field acting on all spins although it is halved for the neighbors of the hole. This term contributes to the Hamiltonian once $\lambda \neq 1$ and explicitly breaks the SU(2) symmetry.  
"


(17) Throughout the paper, we introduce minor adjustments to the style of few sentences. These changes do not influence the message nor the scientific content of this report but they aim solely at improving its readability.


\bibliographystyle{apsrev4-1}
\bibliography{saw}

\end{document}
