\documentclass[12pt, a4paper]{article}
\usepackage[a4paper,margin=1in,footskip=0.25in]{geometry}
 
\usepackage{graphicx}
\usepackage{enumerate}
\usepackage{amsmath}
\usepackage{amssymb}
\usepackage[utf8]{inputenc}
\usepackage{physics}
\usepackage{xcolor}
\usepackage{hyperref}
\usepackage{subfig}
		
\newcommand{\mean}[1]{\langle#1\rangle}

\begin{document}

\section{The~Model}
The $t$-$J$ model Hamiltonian reads,
\begin{equation}
	H = -t\sum_{\mean{i,j}}\left(\tilde{c}_{i\sigma}^\dagger\tilde{c}_{j\sigma} + H.c\right)
	+ J\sum_{\mean{i,j}}\left(\frac{1}{2}\left[S_i^+S_j^- + S_i^-S_j^+\right] + S_i^zS_j^z - \frac{1}{4}\tilde{n}_i\tilde{n}_j\right),
\end{equation}
where $\tilde{c}_{i\sigma}^\dagger = c_{i\sigma}^\dagger(1-n_{i\bar{\sigma}})$ can create electrons only on unoccupied sites. In order to introduce holes and magnons we start with rotation of spins in one sublattice of the system. This results in
\begin{equation}
	H_{\text{rot}} = -t\sum_{\mean{i,j}}\left(\tilde{c}_{i\sigma}^\dagger\tilde{c}_{j\bar{\sigma}} + H.c\right)
	+ J\sum_{\mean{i,j}}\left(\frac{1}{2}\left[S_i^+S_j^+ + S_i^-S_j^-\right] - S_i^zS_j^z\right).
\end{equation}
Rotation transforms Ising Antiferromagnet (IAF) into ferromagnet (FM). This allows for introduction of magnons which can be understood as deviations from ferromagnetic state obtained after rotation. To this end, we use the following transformations,
\begin{equation}
	\begin{aligned}
	\tilde{c}_{i\uparrow}^\dag &= h_i, &\quad \tilde{c}_{i\uparrow} &= h_i^\dag (1 - a_i^\dag a_i), \\
	\tilde{c}_{i\downarrow}^\dag &= h_i a_i^\dag, &\quad \tilde{c}_{i\downarrow} &= h_i^\dag a_i,
	\end{aligned}
\end{equation}
\begin{equation}
	\begin{aligned}
		S_i^+ &= (1 - h_i^\dag h_i)(1 - a_i^\dag a_i)a_i, &\quad S_i^z &= \frac{1}{2} - a_i^\dag a_i - \frac{1}{2}h_i^\dag h_i, \\
		S_i^- &= a_i^\dag (1 - a_i^\dag a_i)(1 - h_i^\dag h_i), &\quad \tilde{n}_i &= 1 - h_i^\dag h_i.
	\end{aligned}
\end{equation}
In fact, one can notice there are two possible IAF states. This leads to two ferromagnetic states. One with all the spins pointing ``up" and the second one with all the spins pointing ``down". After transformation to magnons these states will correspond to state without magnons and state fully occupied by magnons. It is totally up to us to choose which one is which. The above defined transformation sets state with all spins ``up" to vacuum state for magnons and the state with all spins ``down" is transformed onto state fully occupied by magnons. In the end the model (up to the shift by a constant) reads,
\begin{align}
	\mathcal{H} &= \mathcal{H}_{\text{hole}} + \mathcal{H}_{\text{magnon}},
\end{align}
	where,
\begin{equation}
	\begin{split}
	\mathcal{H}_{\text{hole}} &= t \sum_{\mean{i,j}} \left[h_i^\dag h_j \left( a_i + a_j^\dag (1 -  a_i^\dag a_i) \right) + h_j^\dag h_i \left( a_j + a_i^\dag (1 -  a_j^\dag a_j) \right)\right] \\
	&+ \frac{J}{2} \sum_{\mean{i,j}} \left[ h_i^\dag h_i + h_j^\dag h_j - h_i^\dag h_i h_j^\dag h_j - h_i^\dag h_i a_j^\dag a_j - h_j^\dag h_j a_i^\dag a_i \right],
	\end{split}
	\label{eq:ht}
\end{equation}

\begin{equation}
	\begin{aligned}
	\mathcal{H}_{\text{magnon}} &= \frac{J}{2}\sum_{\mean{i,j}} \left[(1 - a_i^\dag a_i)(1 - a_j^\dag a_j)a_i a_j + a_i^\dag a_j^\dag (1 - a_i^\dag a_i)(1 - a_j^\dag a_j) \right]P_i P_j \\
	&+ \frac{J}{2} \sum_{\mean{i,j}} \left[a_i^\dag a_i + a_j^\dag a_j - 2\beta a_i^\dag a_i a_j^\dag a_j \right],
	\end{aligned}
	\label{eq:hj}
\end{equation}
and $P_i = (1 - h_i^\dag h_i)$ projects out sites occupied by a hole. Note that $h_i$ is spinless fermion operator, thus there can be at most one hole per site. The above transformation is exact for spin $S=\frac{1}{2}$ when $\beta = 1$. Also note that definition of spin operator may be different for different spins (e.g. 3-dimensional matrix for $S=1$ compared to 2-dimensional matrix for $S=\frac{1}{2}$). Most of the time notation is abused and one does not care whether e.g. $S^z$ acts on state with spin $S=\frac{1}{2}$ or on state with $S=1$. But in principle there are two different operators, one for each case.


\bibliographystyle{apsrev4-1}
\bibliography{xxz}

\end{document}